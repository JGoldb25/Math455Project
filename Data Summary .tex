\section{Data Summary}
\label{S:4}

\begin{flushleft}
	\subsection{Initial Summary}

\begin{table}[h]
	\centering
	\begin{tabular}{l l l I I}
		\hline
		\textbf{Total Attributes}&\textbf{Numerical}&\textbf{Categorical}&\textbf{Total Instances}&\textbf{Response Variable} \\
		\hline
		13 & 9 & 4 & 517 & area\\
		\hline
	\end{tabular}
	\caption{Broad Level Data Details}
\end{table}

In our analysis, we were looking to see which factors had the greatest impact on how much area would be burned during a forest fire in Portugal. When examining the correlation plot among all our variables, we were not inclined to remove any of the 12 predictors in our data set as none of the predictors exhibited a strong relationship with Area. 

\subsection{Missing Data}
Our data came directly from the UCI Repository. Thus, the data set contained no missing entries, which tremendously helps in making the data analysis that much more accurate. Had we needed to replace any missing values, we would have used multiple imputation techniques to do so.\\
\\
Thus, our data set has 517 complete cases.

\subsection{Complete Cases}
Below are summary statistics on our 9 numerical and 4 categorical variables. Additionally, below you will see our response variable's distribution (which you'll notice is very right skewed), and our correlation chart to examine possible relationships between predictors.

	\begin{table}
	\centering
	\begin{tabular}{l l l I I I I I I I}
		\hline
		\textbf{Index}&\textbf{FFMC}&\textbf{DMC}&\textbf{DC}&\textbf{ISI}&\textbf{temp}&\textbf{RH}&\textbf{wind}&\textbf{rain}&\textbf{area}\\
		\hline
		mean & 90.6 & 110.9 & 547.9 & 9.0 & 18.9 & 44.29 & 4.0 & 0.0 & 12.9 \\ 
		median & 91.6 & 108.3 & 664.2 & 8.4 & 19.3 & 42 & 4 & 0 & 0.5 \\
		std & 5.5 & 64.1  & 248.1 & 4.6 & 5.8 & 16.3 & 1.8 & 0.3 & 63.7 \\
		min & 18.7 & 1.1 & 7.9 & 0 & 2.2 & 15 & 0.4 & 0 & 0 \\
		max & 96.2 & 291.3 & 860.6 & 56.1 & 33.3 & 100 & 9.4 & 6.4 & 1090.8 \\
		range & 77.5 & 290.2 & 852.7 & 52.1 & 31.1 & 85 & 9.4 & 6.4 & 1090.8 \\
		\hline
	\end{tabular}
	\caption{Summary Statistics on Numerical Variables} 
\end{table}

	\begin{table}
	\centering
	\begin{tabular}{l l l I}
		\hline
		\textbf{month}&\textbf{day}&\textbf{Longitudinal X}&\textbf{Latitudinal Y}}\\
		\hline
		Jan: 2 & Mon: 74 & 1: 48 & 1: 0 \\ 
		Feb: 20 & Tue: 64 & 2: 73 & 2: 44 \\
		Mar: 54 & Wed: 54  & 3: 55 & 3: 64  \\
		Apr: 9 & Thu: 61 & 4: 91 & 4: 203 \\
		May: 2 & Fri: 85 & 5: 30 & 5: 125 \\
		Jun: 17 & Sat: 84 & 6: 86 & 6: 74  \\
		Jul: 32 & Sun: 95 & 7: 60 & 7: 0  \\
		Aug: 184 &  & 8: 61 & 8: 1  \\
		Sep: 172 &  & 9: 13 & 9: 6  \\
		Oct: 15 &  &  &  \\
	    Nov: 1 & &  &  \\
		Dec: 9 &  &  &  \\
		\hline
	\end{tabular}
	\caption{Frequency of Categorical Variables} 
\end{table}

\begin{figure}
	\centering
	\includegraphics[width=0.7\linewidth]{"../DENSITY PLOTS/Area Density Plot"}
	\caption{Burnt Area of Forest Fires}
	\label{fig:area-density-plot}
\end{figure}

Here in Table 2, we get a closer look at the summary metrics of the numerical variables in our data set. We can examine several of these statistics in particular to understand the distributions of these data better. For example, most of the predictors have small standard deviations, other than {\it area} and {\it DC}. Moreover, the
ranges for most of these variables are big, especially for {\it area} and {\it DC} once again, and arguably {\it DMC} as well. Additionally, we can tell how far apart the mean and median values are for {\it area} and {\it DC} which hints at these predictors not having skewed distributions. As the density plot for {\it area} is heavily skewed to the right, we will try several transformations to normalize this data - inclduing Box Cox, logarithms, etc.
